\section{Conclusion and Work Plan}
\label{sec:conclusion}
Understanding how day-to-day experiences of discrimination impact psychological state helps us reason about the processes through which such experiences lead to deteriorated mental health. We identified patterns of self-reported and passively sensed behaviors in response to discrimination events in a sample of college students. Our findings illustrate the complexity of response to discriminatory encounters and that it changes over time.%, in a student population. 

Educational institutions are characterized by dominant attitudes and behaviors. Some disciplines are particularly vulnerable to gender, race, and nationality bias, including  engineering \cite{sevo2010bias}, a discipline from which half our participants were drawn.  % (enginering?). I have a couple of hours tomorrow and could help lit search if that is usefu. 

We believe that it is critically important to study these issues in the educational context, a sentiment recently argued in an NSF Dear Colleague Letter encouraging research in sexual harassment and other forms or harrassment in STEM contexts\footnote{\url{https://www.nsf.gov/pubs/2019/nsf19053/nsf19053.jsp}}. The pervasiveness of these experiences was a surprise to our team, and addressing them is critical to creating a diverse and informed workforce. % We were surprised to see 480+ events, but
As Bill and Melinda Gates said in their  recent Annual Letter \footnote{\url{https://www.gatesnotes.com/2019-Annual-Letter}}, data is sexist (and racist) and the biases inherent in the data we collect are necessary, indeed critical to address. This study is a first attempt to do so, and we hope to contribute to the development of this domain as an important topic of study for computational researchers.   %https://www.gatesnotes.com/2019-Annual-Letter?WT.mc_id=02_12_2019_05_AL2019_BG-COM_&WT.tsrc=BGCOM#ALChapter4

A comprehensive change in the way that we study the college student experience. 
Facilitated by ease of capturing real-time student information. 
Innovations
Studying experience at scale
Before-after data around events such as discrimination
Quantification of impact
Allow us to design the most effective interventions and policies

\section{Proposed Study}
\label{sec:study}
\noindent We propose a four-year study that follows 200 students from their first year in the College of Engineering through graduation. Participants in this study will provide massive amounts of raw behavioral data supplemented with survey data that leverages standard scales for measuring mediating factors, exposure to discrimination, and mental and physical health. In addition, participants will provide bi-weekly experience sampling surveys  (EMAs) to answer questions about daily discrimination, daily negative events, and daily changes in anxiety and stress. Details of our approach  follow.

Over the last two years, we have designed, refined, and tested our measures; collected pilot data from 174 students in January-June of 2018; and launched the first year of data collection for the proposed study in April of 2019, with 200 first-year engineers.  Based on compliance with the pilot (which was highest during the first quarter), our plan for future years is to run the study for one quarter per year, and additionally to answer questions about exposure to discrimination and negative events for the entire year.  

Our 2018 pilot study demonstrates the viability and power of our method, both in terms of our ability to collect useful data from participants with high compliance, and our ability to connect that data to interesting results. Our preliminary analysis found that over half (91) of students experienced discrimination, and those students reported 448 distinct incidents. These incidents were associated with reduced sleep, more time on the phone, and changes in physical activity (\ref{itm:rq-behavior}), which lasted 1-2 days after the event (\ref{itm:rq-behavior-size}, as described in Section~\ref{sec:result}. These results are the first to quantify the type, size, and duration of the impact of daily discrimination events on short term behavior. Our study is the first to deploy technology capable of gathering the data necessary to study these types of things.  

We refer to our pilot study throughout this section to illustrate the viability of our approach. However, our primary goal in this section is to document our proposed method. 

\subsection{Method}
\noindent
We propose to collect data from 150 University of Washington Engineering students over four years. This will allow us to study students' experience in a challenging and critical period of their lives. Four years of data is necessary to track long-term changes (an important part of our model) and collecting it is the only way to get a complete picture of factors like retention in major.
Extending the study to four years is feasible within the proposal period  since we are already engaged in year-one data collection.

We expect some attrition, and this is reflected in the budget. However, we hope to have at least 90 students still participating by the end of the study based on our pilot study's retention rate, which was approximately $85\%$. Thus, if we start with 150 students, we expect 132 in year two, 110 in year three, and 92 in year four. \paula{need here or elsewhere to present power analysis? This is a pretty low number for assessment of longitudinal changes and trends.}


\subsubsection{Participants}
\label{sec:study-participants}
We have over-sampled from underrepresented groups in engineering (UREs), including women (gender discrimination is an ongoing problem \cite{johnson2018sexual}), underrepresented minorities (URMs) such as Latinx, African American and Native American students, and first-generation students.  We propose to use snowball sampling and targeted emails to reach our population of interest. \paula{anything more needed to support this sampling strategy? issues of representativeness?}

In addition, we sampled from specific micro-climates of interest (such as STARS), again using targeted emails to reach populations of interest.

We illustrate our ability to recruit a representative sample by summarizing our pilot population in Table~tbl{tab:study-participants}. Note that our pilot study did not focus on engineers. Our year one deployment has already begun, and while we have not completed analysis as of this writing, it has been similarly successful, with participants including $48\%$ women, $14\%$ URMs and half of this year's STARS cohort.  \paula{do we need here more details about micro-climates given that this far we have only noted two and need to more fully define what these are?}

\begin{table}[]
\small
\begin{tabular}{l|c|c|c|c|}
\cline{2-5}
& \multicolumn{2}{c|}{\begin{tabular}[c]{@{}c@{}}  \textbf{Completed Pilot}\end{tabular}} & \multicolumn{2}{c|}{\begin{tabular}[c]{@{}c@{}} \textbf{Dropped Out of Pilot}\end{tabular}} \\ 
\cline{2-5} 
& \textbf{All (N=176)} & \textbf{Engineers (N=73)} & \textbf{All (N=33)} & \textbf{Engineers (N=11)} \\ 
\hline
\multicolumn{1}{|l|}{Women (64\%)} & 114 (54\%) & 41 (20\%) & 19 (9\%) & 7 (3\%)  \\
\hline
\multicolumn{1}{|l|}{\begin{tabular}[c]{@{}l@{}}URM (12\%) \end{tabular}} & 18 (9\%) & 15 (7\%) & 10 (5\%) & 5 (2\%) \\ \hline
\multicolumn{1}{|l|}{First-Generation Students} & 51 (24\%) & 27 (13\%) & 11  (5\%) & 7 (3\%)  \\ \hline
\multicolumn{1}{|l|}{LGBTQIA+ (12\%)} & XX & XX & XXX & XX  \\
\hline 
\end{tabular}
\caption[Pilot - sample breakdown]{Sample breakdown of gender and minority status in our pilot study. Categories are non-independent. Of the 33 who dropped out, 13 did so before the break between quarters, and 20 before post \sandy{"post-quarter"?} questionnaires. URM refers to under-represented minorities (i.e., students who are African-American, Native American, Latinx, and Pacific Islander).
}
\label{tab:study-participants}
\end{table}


\jen{need \%age of first gen students and percentagis for LGBTQIA+}

\subsubsection{Surveys}
Study participants will answer hour-long written questionnaires about their life experiences, self regulation and coping skills, and health behaviors before and after the study (i.e., \textit{pre} and \textit{post} surveys).  They will also fill out twice weekly surveys about their affect, stress, and experiences of unfair treatment (Experiential Momentary Assessment (EMA) surveys).  For two weeks, we will send EMA surveys four times a day to get more detailed information. A summary of survey and EMA measures is shown in Table~\ref{tab:study-surveys}.

Concerns with a study of this intensity are compliance and retention. To help ensure both, we propose to stay in frequent touch with participants and to pay them well for their participation.  Of 209 participants in the pilot study, 176 (84\%)  completed the post-study questionnaire. The  compliance rate for EMA surveys was 85\%. These very high rates give us confidence in the viability of the study method. 

\begin{table}[]
\smaller
\begin{tabular}{p{1.5mm}|p{2.9cm}|l|p{9.3cm}|}
\cline{2-4}
 & \textbf{Measure}   & \textbf{Administration} & \textbf{Scales / Items Included in the Measure} \\ \hline
\multicolumn{1}{|c|}{\multirow{4}{*}{\vspace{-7mm}\rotatebox[origin=c]{90}{Pre or Post}}} & Social Experiences or Perceptions & pre, post & UCLA Loneliness (loneliness) [Russell \citeyear{Russell:1996}], 2-way SSS (social support) [Shakespeare-Finch \citeyear{Shakespeare:2011}]  \\ \cline{2-4} 
\multicolumn{1}{|c|}{} & \multirow{2}{*}{Stress \& Coping} & pre, post & MAAS [Brown, \citeyear{Brown:2003}], ERQ  (emotion regulation) [Gross \citeyear{Gross:2003}], PSS (stress) [Cohen \citeyear{Cohen:1983stress}], BRS  (resilience) [Smith \citeyear{Smith:2008}] \\ \cline{2-4} 
\multicolumn{1}{|c|}{} & Physical \& Mental Health% and Sleep 
& pre, post  & CHIPS [Cohen \citeyear{Cohen:1983positive}] (physical health), CES-D (depression) [Radloff \citeyear{Radloff:1977}], %RSQ \cite{Armey:2009}, 
STAI (anxiety) [Kabacoff \citeyear{Kabacoff:1997}]%, PSQI \cite{Buysse:1989} 
\\ \hline%\cline{2-4} 
\multicolumn{1}{|c|}{\multirow{6}{*}{\vspace{-7mm} \rotatebox[origin=c]{90}{EMA}}} & \multirow{2}{*}{Affect} & daily, weekly & Feeling anxious, depressed, frustrated, overwhelmed, lonely, happy and connected on a scale of 1 (not at all) to 5 (extremely) \\ \cline{2-4} 
\multicolumn{1}{|c|}{} & \multirow{2}{*}{\textit{Unfair Treatment\dag}} %(exposure \& severity) 
& daily, weekly & Unfairly treated because of ancestry or national origins, gender, sexual orientation, intelligence, major, learning disability, education or income level, age, religion, physical disability, height, weight or other aspect of one's physical appearance\\ \hline
\end{tabular}
\caption{Proposed measures in pre- or post-study questionnaires and EMA surveys. The high-level construct related to each measure is shown in parentheses after the acronym. Our proposed measure of daily discrimination is italicized and marked with a cross (\dag). Other scales provide information about context and mediating factors. 
}
\label{tab:study-surveys}
\end{table}


\paragraph{Proposed use of survey data to populate theoretical model.} 
The survey data is the source of our most important \textit{primary stressor} variable, daily discrimination. Participants are asked about this twice-weekly, as follows: ``Did you experience unfair treatment for any of the following reasons?'' A range of reasons will be provided, including gender, race, \etc We will also ask about severity, possible reasons for unfair treatment, and which type of person was responsible (\eg peer, teacher). This will provide the basis, source, and severity for the primary stressor in the model in Figure~\ref{fig:model}.

We will use this data to calculate two additional measures:  \textit{exposure} (any report of unfair treatment qualifies) and \textit{severity} (ratio of total reports to total available responses, \ie number of times the question was answered over the course of the study). 

Survey data will also be used to measure \textit{mediating factors}, including resilience, social support, chronic discrimination and negative events (left-most box in Figure~\ref{fig:model}). We propose to use a  \textit{pre/post} scale that asks about discrimination over the lifetime, year, and quarter, along with other negative events. We also use standard, well-tested psychological scales to measure factors like Resilience and Social Support (see Table~\ref{tab:study-surveys}). 

\textit{Long-term behavior outcomes} will come from a combination of survey data about mental and physical health (as assessed at the end of the study) and institutional data provided by the university (we have access to data about grades and retention). \textit{Context} will come from survey data.

\subsubsection{Short-Term Behavior Data}
We will collect short-term behavior data (orange box on right side of model in Figure~\ref{fig:model}\sandy{I don't see this on figure 1}) using a combination of custom phone software and a Fitbit. We propose to give participants a Fitbit~Flex~2, which records the number of steps taken and per-minute sleep status (\eg asleep or awake). The phone software we will use is the AWARE Framework \cite{Ferreira:2015}. AWARE has been tested in many studies, including our own pilot study, and provides an excellent, stable option for such studies. It can collect a wide range of passive data, including  
location; phone screen status; call logs for incoming, outgoing and missed calls; and activity information (\eg walking, running, or still) inferred by the phone. 
\tbl{tab:study-sensors} summarizes the sensors we propose to collect and proposed details of their collection (\eg sampling rate) based on what we learned in our pilot study. 

\paragraph{Proposed use of short-term behavior data.} We propose to operationalize the short-term behavior data, as summarized in Table~\ref{tab:study-sensors}.  We will extract behavior features using  the AWARE feature extraction library \citep{Chikersal:2019}. We calculate features daily and for four different parts of the day: night (12am-6am), morning (6am-12pm), afternoon (12pm-6pm), and evening (6pm-12am) in an approach modeled on \citep{wang2014studentlife}'s study, which connected behavior to stress on a similar data set. 

We will group the features into the five categories that the literature suggests are relevant to stress (see below). \paula{there are other behaviors more or additionally linked to stress such as substance use, social media, etc. The table below includes passive data. Is there value in noting here that we are also including self-report behavioral data?} We focus on stress literature because of the lack of information about which of these behaviors will be impacted by  exposure to discrimination. However, since discrimination is known to  adversely affect mental and physical health, and to cause distress (\eg \cite{Ong:2009}), focusing on stress-related behaviors makes sense as a starting place. Our pilot data analysis provides further evidence for this choice, as described in Section~\ref{sec:result}.

\begin{description}
\item[Physical Activity.]     Higher levels of physical activity are correlated with fewer symptoms of anxiety and depression \citep{Stephens:1988}, as well as lower levels of emotional distress \citep{Steptoe:1996}. Moreover, past work on mobile health sensing has successfully used features based on inferred activity, \eg to predict depression in students \citep{Wang:2018} or relapse in schizophrenic patients \citep{Wang:2016}. 
\item[Phone Usage.] Distraction, an emotion-regulation strategy to reduce distress and negative feelings \citep{Sheppes:2011}, can manifest itself in the form of excessive or purposeless phone screen time. In fact, phone screen overuse is linked to depression and anxiety in college students \citep{Demirci:2015}. Encouragingly, patterns of phone screen time, particularly in relation to location of use, have been previously used as depression symptoms \citep{Wang:2018}. \paula{only below is it clear that phone usage means screen only, not calling--see if I got this right}
\item[Social Interaction.] Social support and interaction are key to psychological health and well-being \citep{Kawachi:2001}. Unsurprisingly, mental health problems, such as depression, are inversely related to quality and quantity of social interactions \citep{Nezlek:1994}. Moreover, social-support seeking is a common strategy people use to cope with distress \citep{Carver:1997}.  In terms of social context, previous research has used information on nearby Bluetooth devices (\eg \cite{wang2014studentlife}), which can potentially encompass friendship information as suggested in \citep{Eagle:2009}. Social interactions have been used as indicators of mental health \citep{Wang:2016}.  \paula{so soc interactions are info about calls only, yes? does it pick up texting? emailing, etc?}
Discriminatory encounters can initially lead to increased calls as people seek support. However, when depressive symptoms increase, social participation might drop (\eg social withdrawal \citep{Girard:2014}) . 
\item[Mobility.] Mental health conditions, such as depression or anxiety, that are characterized by avoidance behaviors can potentially impact mobility patterns. \paula{provide more info about what "variety of activities" means in Table 3; this will help distinguish mobility and activity and the differing interpretations of these data} There is indeed evidence connecting people's mobility to the severity of depressive symptoms \citep{Saeb:2015, Saeb:2016} and their levels of anxiety \citep{Huang:2016}. 
\item[Sleep.] There is significant comorbidity between sleep problems and a number of mental health complications, including depression \citep{Vandeputte:2003}, anxiety \citep{Morrison:1992}, and malconduct \citep{Papadimitriou:2005}. Sleep detection using wearable and mobile sensors has been the topic of mobile health research (\eg \cite{Min:2014}), and sleep monitors are commercially available, \eg through Fitbit. In relation to mobile sensing of mental health, \citet{wang2014studentlife} and \citet{Wang:2018} have used measures of sleep to model academic performance and levels of depression in college students. 
\end{description}

\vspace{1em}
\noindent
To summarize, the behaviors that we might expect based on the literature include reduced physical activity, increased phone screen time, reduced phone calls, reduced mobility, and increased sleep disruptions. These five areas of potential behavior change form the basis for our features, which are summarized in Table~\ref{tab:study-sensors}.

\begin{table}[]
\centering
\smaller
\begin{tabular}{|l|l|l|l|p{5.5cm}|}
\hline
\textbf{Relevant Behavior}         & \textbf{Sensor} & \textbf{Source}        & \textbf{Sampling}            & \textbf{Information Collected}                                                 \\ \hline
\multirow{2}{*}{Physical Activity} & Step            & Fitbit                 & 1 sample per min             & Number of steps                                                                \\ \cline{2-5} 
                                   & Activity        & \multirow{4}{*}{AWARE} & 1 sample per 5 min           & Type of activity: walking, running, on bicycle, in vehicle, still, unknown     \\ \cline{1-2} \cline{4-5} 
\multirow{2}{*}{Phone Usage}                        & Screen          &                        & \multirow{2}{*}{event-based} & Screen status (locked, unlocked, off, and on) events                           \\ \cline{1-2} \cline{5-5} 
\multirow{2}{*}{Social Interactions} & Call            &                        &                              & Time and duration of incoming, outgoing, and missed calls                      \\ \cline{1-2} \cline{4-5} 
\multirow{2}{*}{Mobility}               & Location        &                        & 1 sample per 10 min          & GPS latitude, longitude, altitude                                              \\ \cline{2-2} \cline{4-5}
& Activity & & 1 sample per 5 min & Variety of activities \\ \hline
%Mobility               & Location        &                        & 1 sample per 10 min          & GPS latitude, longitude, altitude\\ \hline
\multirow{2}{*}{Sleep }                             & Sleep           & Fitbit                 & 1 sample per min             & Duration and onset of sleep, minutes to fall sleep, of awake, and after wakeup \\ \hline
\end{tabular}

\caption[Sensors]{Sensor data collected by AWARE and used in our analysis.}
\label{tab:study-sensors}
\end{table}

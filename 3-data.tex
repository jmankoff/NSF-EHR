\section{Proposed Study}
\label{sec:study}
\noindent Our proposal is to conduct a four year study, following 150 students from their first year in the College of Engineering through graduation. Participants in this study will provide massive amounts of raw behavioral data supplemented with survey data that leverages standard scales for measuring mediating factors, exposure to discrimination, and menttal and physical health. In addition, participants will provide bi-weekly experience sampling surveys  (EMAs) to answer questions about daily discrimination, daily negative events, and daily changes in anxiety and stress. More details on all of this is provided below.

Over the last two years, we have designed, refined, and tested our measures, collected pilot data from 174 students, and launched the first year of data collection for this study with 200 first year engineers. Our pilot ran for two quarters, and based on compliance with the pilot (which was highest during the first quarter), our plan for future years is to have participants provide detailed data for one quarter per year, and additionally to answer questions about exposure to discrimination and negative events for the entire year.  

Our pilot study demonstrates the viability and power of our method, both in terms of our ability to collect data from participants with high compliance that is useful, and our ability to connect that data to interesting results. Although our analysis is by no means complete, we have been able to link daily discrimination to next day behavior, specifically screen use, sleep, and activity (relates to \ref{itm:rq-behavior}) and have evidence suggesting that this lasts only for 1-2 days (relates to \ref{itm:rq-behavior-size}), as described in Section~\ref{sec:result}. 

We refer to our pilot study throughout this section to illustrate the viability of our our approach, but the primary goal of this section is to document our proposed method. 

\subsection{Method}
\noindent
We propose to collect data from 150 UW Engineers over four years at a large public university. This will allow us to study students' experience in one of the challenging and critical periods of life. Four years of data is necessary to track long-term changes (an important part of our model), and is the only way to get a complete picture of things like retention in major.
Extending the study to four years is feasible within the period of the proposal, since we are already engaged in year one data collection.

We expect some attrition, as is reflected in the budget, but hope to have at least 90 students still participating by the end of the study based on our plot study's retention rate, which was approximately $85\%$ retention rate. Thus, if we start with 150 students, we would expect 132 in year two, 110 in year three, and 92 in year four. 


\subsubsection{Participants:}
\label{sec:study-participants}
We are particularly interested in sampling from URE including women (gender discrimination is an ongoing problem \cite{johnson2018sexual}), people of color (particularly under-represented minorities (URMs) such as Latinx, African-American and Native American students), and first generation students.  We thus propose to use snowball sampling and targeted emails to reach our population of interest.

In addition, we will work hard to sample from specific  micro-climates of interest (such as STARS), again using targeted emails to reach populations of interest.

We illustrate our ability to recruit a representative sample by summarizing our pilot population in Table~tbl{tab:study-participants}. Note that our pilot study did not focus on engineers. Our year one deployment as already begun, and while we have not completed analysis as of this writing, it has been similarly successful with $48\%$ women, $14\%$ URMs and half of this year's STAR's cohort enrolled in the study. 

\begin{table}[]
\small
\begin{tabular}{l|c|c|c|c|}
\cline{2-5}
& \multicolumn{2}{c|}{\begin{tabular}[c]{@{}c@{}}  \textbf{Completed Study}\end{tabular}} & \multicolumn{2}{c|}{\begin{tabular}[c]{@{}c@{}} \textbf{Dropped Out}\end{tabular}} \\ 
\cline{2-5} 
& \textbf{All (N=176)} & \textbf{Engineers (N=73)} & \textbf{All (N=33)} & \textbf{Engineers (N=11)} \\ 
\hline
\multicolumn{1}{|l|}{Women (64\%)} & 114 (54\%) & 41 (20\%) & 19 (9\%) & 7 (3\%)  \\
\hline
\multicolumn{1}{|l|}{\begin{tabular}[c]{@{}l@{}}URM (12\%) \end{tabular}} & 18 (9\%) & 15 (7\%) & 10 (5\%) & 5 (2\%) \\ \hline
\multicolumn{1}{|l|}{First Generation Students} & 51 (24\%) & 27 (13\%) & 11  (5\%) & 7 (3\%)  \\ \hline
\multicolumn{1}{|l|}{LGBTQIA+ (12\%)} & XX & XX & XXX & XX  \\
\hline 
\end{tabular}
\caption[Pilot - sample breakdown]{Sample breakdown in terms of gender and minority status in our pilot study. Categories are non-independent. Of the 33 who dropped out, 13 did so before the break between quarters, and 20 before post questionnaire. URM refers to under-represented minorities (African-American students, Native American students, Latinx students, and Pacific Islander students).
}
\label{tab:study-participants}
\end{table}


\jen{need \%age of first gen students and percentagis for LGBTQIA+}

\subsubsection{Surveys}
Participants in our study will answer hour-long questionnaires about their life experiences, self regulation and coping skills, health behaviors, before and after the study (\textit{pre} and \textit{post} surveys).  Participant will also answer  twice weekly surveys about their affect, stress, and experiences of unfair treatment (EMA surveys).  For two weeks, we will send EMA surveys four times a day to get more detailed information. A summary of survey and EMA measures is shown in Table~\ref{tab:study-surveys}.

A concern with a study of this intensity is compliance during the study, as well as retention throughout the study. To help ensure compliance, we propose to stay in frequent touch with participants and to pay them well for their participation.  Of 209 participants in the pilot study, 176 (84\%)  completed the post questionnaire. The  compliance rate for EMA surveys was 85\%. These are very high compliance rates and give us confidence in the viability of the study method. 

\begin{table}[]
\smaller
\begin{tabular}{p{1.5mm}|p{2.9cm}|l|p{9.3cm}|}
\cline{2-4}
 & \textbf{Measure}   & \textbf{Administration} & \textbf{Scales / Items Included in the Measure} \\ \hline
\multicolumn{1}{|c|}{\multirow{4}{*}{\vspace{-7mm}\rotatebox[origin=c]{90}{Pre or Post}}} & Social Experiences or Perceptions & pre, post & UCLA Loneliness(loneliness) [Russell \citeyear{Russell:1996}], 2-way SSS (social support) [Shakespeare-Finch \citeyear{Shakespeare:2011}]  \\ \cline{2-4} 
\multicolumn{1}{|c|}{} & \multirow{2}{*}{Stress \& Coping} & pre, post & MAAS [Brown, \citeyear{Brown:2003}], ERQ  (emotion regulation) [Gross \citeyear{Gross:2003}], PSS (stress) [Cohen \citeyear{Cohen:1983stress}], BRS  (resilience) [Smith \citeyear{Smith:2008}] \\ \cline{2-4} 
\multicolumn{1}{|c|}{} & Physical \& Mental Health% and Sleep 
& pre, post  & CHIPS [Cohen \citeyear{Cohen:1983positive}] (physical health), CES-D (depression) [Radloff \citeyear{Radloff:1977}], %RSQ \cite{Armey:2009}, 
STAI (anxiety) [Kabacoff \citeyear{Kabacoff:1997}]%, PSQI \cite{Buysse:1989} 
\\ \hline%\cline{2-4} 
\multicolumn{1}{|c|}{\multirow{6}{*}{\vspace{-7mm} \rotatebox[origin=c]{90}{EMA}}} & \multirow{2}{*}{Affect} & daily, weekly & Feeling anxious, depressed, frustrated, overwhelmed, lonely, happy and connected on a scale of 1 (not at all) to 5 (extremely) \\ \cline{2-4} 
\multicolumn{1}{|c|}{} & \multirow{2}{*}{\textit{Unfair Treatment\dag}} %(exposure \& severity) 
& daily, weekly & Unfairly treated because of ancestry or national origins, gender, sexual orientation, intelligence, major, learning disability, education or income level, age, religion, physical disability, height, weight or other aspect of one's physical appearance\\ \hline
\end{tabular}
\caption{Proposed measures in pre or post questionnaires and EMA surveys. The high level construct related to each measure is provided after the acronym and within parentheses. Our proposed measure of daily discrimination is italicized and marked with a cross (\dag). Other scales are used to provide information about context and mediating factors. 
}
\label{tab:study-surveys}
\end{table}


\paragraph{Proposed use of survey data to populate theoretical model} 
The survey data is the source of our most important \textit{primary stressor } variable, daily discrimination. Participants are asked about this twice-weekly as follows: ``Did you experience unfair treatment for any of the following reasons.'' A range of reasons will be provided including gender, race, \etc. We will also ask about severity, possible reasons for unfair treatment, and which type of person was responsible (\eg peer, teacher). This will provide the basis, source, and severity for the primary stressor in the model presented in Figure~\ref{fig:model}.

We will use this data to calculate two additional measures:  \textit{exposure} (any report of unfair treatment qualifies) and \textit{severity} (ratio of total reports to total available responses, \ie number of times the question was answered over the course of the study). 

Survey data will also be used to measure \textit{mediating factors} including resilience, social support, chronic discrimination and negative events (left-most box in Figure~\ref{fig:model}). We propose to use a  \textit{pre/post} scale that asks about discrimination over the lifetime, year, and quarter, along with other negative events. We also use standard well-tested psychological scales to measure things like Resilience and Social Support (see Table~\ref{tab:study-surveys}). 

\textit{Long term behavior outcomes} will come from a combination of survey data about mental and physical health (ass assessed at the end of the study) and institutional data provided by the university (we have access to data about grades and retention). \textit{Context} will come from the survey data.

\subsubsection{Short-term behavior data}
We will collect the short term behavior data (orange box on right side of model in Figure~\ref{fig:model}) using a combination of custom phone software and a fitbit. We propose to give participants a Fitbit~Flex~2, which records the number of steps taken and per-minute sleep status (\eg asleep or awake). The phone software we will use is the Aware Framework \cite{Ferreira:2015}. Aware has been tested in many studies, including our own pilot study, and an excellent, stable option for such studies. It can collect a wide range of passive data including  
location, phone screen status, call logs for incoming, outgoing and missed calls, and activity information (\eg walking, running, or still) inferred by the phone. 
\tbl{tab:study-sensors} summarizes the sensors we propose to collect and proposed details of their collection (\eg sampling rate), based on what we learned in our pilot study. 

\paragraph{Proposed use of short-term behavior data} We propose to operationalize the short-term behavior data as summarized in Table~\ref{tab:study-sensors}.  We will extract behavior features using  the AWARE feature extraction library \citep{Chikersal:2019}. We propose to calculate features daily and for four different parts of the day: night (12am-6am), morning (6am-12pm), afternoon (12pm-6pm), and evening (6pm-12am), in an approach modeled on \citep{wang2014studentlife}'s study which connected behavior to stress on a similar data set. 

We will group the features into  the five categories that the literature suggests are relevant to stress. We focus on stress literature because of the lack of information about which of these behaviors will be impacted by  exposure to discrimination. However, since discrimination is known to  impacts mental and physical health, and to cause distress (\eg \cite{Ong:2009}), focusing on stress-related behaviors makes sense as a starting place. Our pilot data analysis provides further evidence for this choice, as described in Section~\ref{sec:result}.

\begin{description}
\item[Physical Activity]     Higher levels of physical activity are correlated with fewer symptoms of anxiety and depression \citep{Stephens:1988} as well as lower levels of emotional distress \citep{Steptoe:1996}. Moreover, past work on mobile health sensing has successfully used features based on the inferred activity, \eg to predict depression in students \citep{Wang:2018} or relapse in schizophrenic patients \citep{Wang:2016}. 
\item[Phone Usage] Distraction, an emotion-regulation strategy to reduce distress and negative feelings \citep{Sheppes:2011}, can manifest itself in the form of excessive or purposeless phone use. In fact, phone overuse is linked to depression and anxiety in college students \citep{Demirci:2015}. Encouragingly, patterns of phone use, particularly in relation to location of use, have been previously used as depression symptoms \citep{Wang:2018}.
\item[Social Interaction] Social support and interaction is key to psychological health and well-being \citep{Kawachi:2001}. Unsurprisingly, mental health problems, such as depression, are inversely related to quality and quantity of social interactions \citep{Nezlek:1994}. Moreover, social support seeking is a common strategy people use to cope with distress \citep{Carver:1997}.  In terms of social context, previous research has used information on nearby Bluetooth devices (\eg \cite{wang2014studentlife}), which can potentially encompass friendship information as suggested in \citep{Eagle:2009}. Social interactions have been used as indicators of mental health \citep{Wang:2016}.  
Discriminatory encounters can initially lead to increased calls as people seek support. However, when the depressive symptoms (\eg withdrawal \citep{Girard:2014}) increase, social participation might drop. 
\item[Mobility] Mental health conditions, such as depression or anxiety, that are characterized by avoidance behaviors can potentially impact mobility patterns. There is indeed evidence connecting people's mobility to the severity of depressive symptoms \citep{Saeb:2015, Saeb:2016} and the levels of anxiety \citep{Huang:2016}. 
\item[Sleep] There is significant comorbidity between sleep problems and a number of mental health complications including depression \citep{Vandeputte:2003}, anxiety \citep{Morrison:1992}, and malconduct \citep{Papadimitriou:2005}. Sleep detection using wearable and mobile sensors has been the topic of mobile health research (\eg \cite{Min:2014}) and commercially available, \eg through Fitbit. In relation to mobile sensing of mental health, \citet{wang2014studentlife} and \citet{Wang:2018} have used measures of sleep to model academic performance and levels of depression in college students. 
\end{description}

\vspace{1em}
\noindent
To summarize, the behaviors that we might expect based on the literature include reduced physical activity, increased phone screen time, reduced phone calls, reduced mobility, and increased sleep disruptions. These five areas of potential behavior change form the basis for our features, which are summarized in Table~\ref{tab:study-sensors}.

\begin{table}[]
\centering
\smaller
\begin{tabular}{|l|l|l|l|p{5.5cm}|}
\hline
\textbf{Relevant Behavior}         & \textbf{Sensor} & \textbf{Source}        & \textbf{Sampling}            & \textbf{Information Collected}                                                 \\ \hline
\multirow{2}{*}{Physcial Activity} & Step            & Fitbit                 & 1 sample per min             & number of steps                                                                \\ \cline{2-5} 
                                   & Activity        & \multirow{4}{*}{AWARE} & 1 sample per 5 min           & Type of activity: walking, running, on bicycle, in vehicle, still, unknown     \\ \cline{1-2} \cline{4-5} 
\multirow{2}{*}{Phone Usage}                        & Screen          &                        & \multirow{2}{*}{event-based} & Screen status (locked, unlocked, off, and on) events                           \\ \cline{1-2} \cline{5-5} 
\multirow{2}{*}{Social Interactions} & Call            &                        &                              & Time and duration of incoming, outgoing, and missed calls                      \\ \cline{1-2} \cline{4-5} 
\multirow{2}{*}{Mobility}               & Location        &                        & 1 sample per 10 min          & GPS latitude, longitude, altitude                                              \\ \cline{2-2} \cline{4-5}
& Activity & & 1 sample per 5 min & Variety of activities \\ \hline
%Mobility               & Location        &                        & 1 sample per 10 min          & GPS latitude, longitude, altitude\\ \hline
\multirow{2}{*}{Sleep }                             & Sleep           & Fitbit                 & 1 sample per min             & Duration and onset of sleep, minutes to fall sleep, of awake, and after wakeup \\ \hline
\end{tabular}

\caption[Sensors]{Sensor data collected by AWARE and used in our analysis.}
\label{tab:study-sensors}
\end{table}

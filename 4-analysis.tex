\section{Proposed work: Increasing the Specificity and Robustness of Analysis}
The pilot work just shown is exciting and compelling, but it is not sufficient to solve the problems this proposal sets out to address. 



\subsection{Diving Deeper into the data}
%\paragraph{Interviews about Discrimination:} 
In addition, there are specific topics that require follow up data collection. For example, we propose to conduct interviews that can help to enrich our understanding of students’ experience of unfair treatment. These interviews will in turn guide both quantitative data analysis and further refinements to our data collection procedures. We are currently refining an interview protocol based on pilot interviews as well as our reading of the literature. We hope these interviews will help to answer several questions that can guide our data analysis including:
What reactions do students remember having? How does this relate to what we observe in the data itself? 
What stakeholders and settings are most important to understanding the experience of discrimination? 
Answers to these questions can guide (1) what we focus on quantifying in the data analysis (2) how social science researchers who study discrimination might want to design their study protocols to capture key facets of discrimination.


\subsection{Intervention}
It is our goal not only to describe the student experience, but also to help improve it. Thus, starting in year two of this proposal (years three/four of our data collection) it is our goal to start exploring the potential to intervene in the student process. Our plan is to explore interventions that impact both internal and external factors. Specifically, for internal interventions, we will explore the value of teaching a variety of methods for increasing resilience and other protective factors identified through analysis of the collected data from our pilot study and year one.

For external interventions, we plan to study the impact of STARS in more depth by following a matched sample of students in and not in STARS with similar demographics. We will also work to assess specific things about STARS driven by our research. For example, based on our data about gender, educating students about how to act toward each other, or modifying group membership or training groups to ensure fairer treatment, are both examples of external work that may be able to help address discrimination. These interventions essentially help to create new micro climates that are more protective/supportive. 
%Extended Data Analysis: From Description to Prediction
%Our proposed work has allowed us to capture real-time information about the student experience at scale. Our data analysis goals will operate at several scales (descriptive, analytic, and predictive).

%\subsection{Further analysis of the student experience}
%Descriptive Goals: From a descriptive perspective, our goal is to develop specific descriptive analysis that have immediate value for policy. For example, as we expand our sample over years and across different groups, we should be able to use this to drive answers to questions such as 

%Whether direct admit students benefit more when they reach critical mass (engineering is rapidly increasing the number of direct admit students, so we have a natural experiment in place that lets us compare across years).

Can we differentiate individual versus system differences

Under what circumstances do difficult events cascade into a larger set of troublesome issues for students? In our data, some people seem to bounce back while others don’t. What differentiates these groups of students?

\jen{ say anything about prediction? I'm not sure we have enough analytics for that in this proposal, probably a different proposal?}

%Move from a detection model to a prediction model: The descriptive and analytic goals that we are putting together are contributions in their own right, but taken together they set the scene for something equally important: Predictive work. We have some preliminary evidence that the data supports prediction of things such as increases in depression, XYZ. Some specific goals for prediction:
%Can we predict problems before they occur which can support intervention in the future?
%Can we predict things that are onerous to collect with high enough accuracy that we could reduce the EMAs (e.g. experiences of discrimination)?
%Xxx what else 
%Summarize the importance and impact that prediction could have….


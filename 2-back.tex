\section{Background and Related Work}
\label{sec:back}

\label{sec:back-discrimination}
\noindent 
Discrimination, defined as adverse behaviors, negative judgment, or unfair treatment towards a person, is an uncontrollable and unpredictable stressor, and as such severely influences health and well-being \citep{Williams:2009}. However, there are also risk and protective factors that can explain why similar experiences impact people differently \citep{SeridoAlmeidaWethington:2004}.

The impact of discrimination, in terms of the response it elicits and  health outcomes, can thus be studied within a general stress and coping framework \citep{Pearlin:1999}. 
Ong \etal (\citeyear{Ong:2009}) apply this framework to  study  response to incidents of discrimination, using it to account for the variability in different people's stress responses based on the previous and current context of their life. They developed a theoretical model relating chronic and daily exposure to discrimination to stress based on Pearlin \etal's stress and coping framework \citep{Pearlin:1999}. In a  study of  174 African American doctoral students they demonstrated the relationship between both chronic and daily discrimination to distress, and show that chronic exposure to discrimination exacerbates reaction to daily discrimination incidents (but not other negative events). Since then, other works have demonstrated a clear relationship between discrimination and stress (\eg \cite{Pieterse:2007}). 
Because of discrimination's links to stress, it is not surprising that it can impact both physical and psychological wellbeing, as summarized next:

\paragraph{Physical Well-being:}
\label{sec:back-discrimination-physical}
When discrimination triggers physiological stress responses it can cause
(\eg heightened blood pressure, heart rate, and cortisol secretions \citep{Brondolo:2008, Steffen:2003, Smart:2010}), which can lead to serious conditions such as heart disease (\eg \cite{Marshall:1997, Cohen:1994}). Should it happen repeatedly, discrimination increases reactivitity to stressful situations \citep{GuyllMatthewsBrom-berger:2001} and weakens the body's protective resources, thus increasing the risk of illness similar to other forms of cumulative stress \citep{GeeSpencerChenTakeuchi:2007}. Additionally, discrimination is directly correlated with more unhealthy behavior (\eg smoking, drinking, substance use \citep{LandrineKlonoff:1996, MartinTuchRoman:2003}). 

\paragraph{Psychological Well-being:}
\label{sec:back-discrimination-mental}
The association between exposure to discrimination and mental health is well supported by empirical evidence \citep{Pascoe:2009} as well as large scale population studies \citep{Kessler:1999}. Not only is discrimination directly associated with higher levels of depression, anxiety, and psychological distress in general, it is negatively correlated with identifiers of healthy psyche such as positive affect \citep{Schmitt:2014}. The magnitude of the associations is larger for negative health outcomes (\eg depression \citep{Schmitt:2014}) and is comparable to major stressors such as sexual assault or combat experience \citep{Kessler:1999}. 
Consistent with Ong \etal's (\citeyear{Ong:2009}) application of the stress and coping framework to discrimination, responses are different depending on things like prior exposure  \citep{Kessler:1999}. 

\subsection{Open Questions}
\noindent The relationships documented in the literature are summarized in Figure~\ref{fig:model}. Although there is a great deal of evidence that both daily and cumulative discrimination are linked to stress, and thus impact both physical and psychological wellbeing.  However, a unified model that relates short term behavior to long term outcomes is still lacking. Although Ong \etal (\citeyear{Ong:2009}) propose multiple models, their work focused only on self reported measures. In addition, it is not yet clear how discrimination relates specifically to academic outcomes such as grades and retention. Finally, no prior work has quantified the impact of risk factors and protective factors on these relationships. 
A better understanding of the relationship between short term changes and long term outcomes can help us to design better interventions and improve our ability to model the overall effects of discrimination.
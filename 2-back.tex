\section{Background and Related Work}
\label{sec:back}
\noindent We describe stress and coping framework as the conceptual model for the study of health outcomes associated with discrimination and review some of the evidence regarding the deleterious impact of unfair treatment on physical and mental health. We establish the feasibility of our proposed approach for the study of discrimination by reviewing research on applications of mobile sensing in mental health and highlight that phone data can capture behavioral correlates of various psychological states related to discrimination.
%\jm{this text says \textit{that} you will talk about a bunch of things, but not much about \textit{what} you will say. With the same number of words (approx) can you provide more of a `hook'?} \jm{In addition, you need to be clearer about what you're referencing, which I've marked in several places below. Slightly more detail about referenced work, and clear signaling about which references are discrimination specific is needed}
%\yasaman{Jen is it better in communicating `what' will be covered in the section?}\jm{yes :)}

\subsection{Discrimination and Well-being}
\label{sec:back-discrimination}
\noindent 
Discrimination, defined as adverse behaviors, negative judgment, or unfair treatment towards a person, is an uncontrollable and unpredictable stressor, and as such severely influences health and well-being \cite{Williams:2009}. %\jm{What are you really trying to say with these words? Why is it uncontrollable and unpredictable?} \yasaman{uncontrollable and unpredictable stressors are the kind of stressors that most negatively influence mental health.}
Its impact, in terms of the response it elicits and the health outcomes, can thus be studied within the general stress and coping framework \cite{Pieterse:2007}. %\jm{is there only one? What is this framework?} \yasaman{I am unaware of others but cannot claim they do not exist}
This framework describes a stress process model for the impact of stressors on psychological well-being and how coping, social support, and mastery resources moderate the response. According to this framework, not only are some people more likely to experience discrimination (differential exposure), but also they would react to the experience more strongly (differential reactivity). Moreover, some stressors multiply and generate other stressors (stress proliferation) \cite{Pearlin:1999}. %\jm{I don't think I understand te framewrok yet, you've just told me some implications of it, not what it's about}\yasaman{I added a brief explanation of the framework as far as I understand it}
Within this framework, we can more holistically study response to incidents of discrimination in that we have a way of accounting for the variability in different people's responses based on the previous and current context of their life \cite{Ong:2009}. For example, we can represent this context in terms of risk and protective factors and explain why similar experiences impact people differently \cite{SeridoAlmeidaWethington:2004}. %\yasaman{TO-DO need to double check some of the references}

Below, we describe the existing research on the impact of discrimination on physical and mental health within the stress and coping framework. We highlight the paucity of knowledge explaining the long term mental health disparities associated with discrimination, particularly in terms of short-term differences in behaviors. %\ycomment{TO-DO CRUNCH explain why the knowledge at behavioral level is important, \eg because it would be useful for intervention design.}

\paragraph{Physical Well-being:}
\label{sec:back-discrimination-physical}
%%% OUTLINE %%%
% discuss what is known of the pathways / mechanisms (\ie models that describe the pathway from short-term to long-term outcomes), particularly in relation to risk and protective factors
Acting as a stressor, discrimination triggers physiological stress responses %\yasaman{TO-DO address Jen's comments}\jm{Do we know that discrimination does this, or just that stress does this? If the former, I'd change the start of this sentence to `When discrimination triggers... it can cause ... '} 
(\eg heightened blood pressure, heart rate, and cortisol secretions \cite{Brondolo:2008, Steffen:2003, Smart:2010}) that can lead to serious conditions such as heart disease \cite{Marshall:1997, Cohen:1994}. Should it happen repeatedly, discrimination increases reactivitity to stressful situations \cite{GuyllMatthewsBrom-berger:2001} (differential reactivity) and weakens body's protective resources, thus increasing the risk of illness similar to other forms of cumulative stress \cite{GeeSpencerChenTakeuchi:2007} (stress proliferation). %\yasaman{TO-DO address Jen's comments}\jm{Here and for differential reactivity, is the reference about discrimination specifically or just this term? Need to be precise which you're referencing, if the latter, pt it inside these parens, if the former, I'd add slightly more detail such as... in one study of X people experiencing discrimination \cite{}}. 
Additionally, discrimination is directly correlated with more unhealthy behavior (\eg smoking, drinking, substance use) \cite{LandrineKlonoff:1996, MartinTuchRoman:2003}. %\yasaman{TO-DO need to double check some of the references}

%\yasaman{TO-DO address Jen's comments}\jm{I'd love to see a paragraph here saying what is quantitatively known about the impact of discrimination on these physical well being factors. \% increase in various things, for example. You can also highlight what is not known, and make the connection to lack of certain methods (e.g. continuous passive data collection)}

\paragraph{Psychological Well-being:}
\label{sec:back-discrimination-mental}
%%% OUTLINE %%%
%discuss what is known of the pathways / mechanisms (\ie models that describe the pathway from short-term to long-term outcomes), particularly in relation to risk and protective factors. Explain what is not known and set the stage for for introducing the contributions you are hoping to make
%\yasaman{TO-DO address Jen's comments}\jm{This paragraph does a much better job of signalling clearly which references are about discrimination. I think I'd use a paralel construction to the previous subsection though -- talk about the mechanisms first, then a paragraph on what has been quantified about impact}. 
%
%Paula : I think... 
The association between exposure to discrimination and mental health is well supported by empirical evidence \cite{Pascoe:2009} as well as large scale population studies \cite{Kessler:1999}. Not only is discrimination directly associated with higher levels of depression, anxiety, and psychological distress in general, it is negatively correlated with identifiers of healthy psyche such as positive affect \cite{Schmitt:2014}. The magnitude of the associations is larger for negative health outcomes (\eg depression) \cite{Schmitt:2014} and is comparable to major stressors such as sexual assualt or combat experience \cite{Kessler:1999}. Consistent with stress and coping framework, there are differences in exposure and reactivity to discrimination \cite{Kessler:1999}. %\yasaman{TO-DO address Jen's comments}\jm{explain more}. 
\citet{Ong:2009} also provide evidence for stress proliferation, \eg in that people who experience discrimination are more likely to report other daily stressors. However, it is not yet clear what form the differential reactivity takes or what pathways lead to stress proliferation. More generally, \textit{what about exposure to discrimination leads to higher depression, anxiety, and distress in the long run?} While studies such as \citep{Ong:2009} does show the relationship exists between daily experience of discrimination and health outcomes, it is unclear what the short-term impact is and how that translates to large disparities over time. %Moreover, while we can explain the differences in people's responses to discrimination with the help of risk and protective factors, it is unclear in \textit{what way these factors intensify or buffer} the negative consequences of discrimination. Should we know more about the short-term impact, \eg in terms of changes in behavior, and how that varies according to risk and protective factors, we can better model the pathways connecting individuals' experiences with discrimination to their mental health. \yasaman{we ended up not looking at this for now so I would rather removing it}

%\yasaman{do we need a subsection on the methods re study of discrimination? our story is that the knowledge does not exist because not the right methods have been at researchers' disposal. But it can be out of scope for the present work to talk about methodology in the study of discrimination and bias. A good resources to check is: handbook of prejudice, stereotyping, and discrimination.}\jm{see my comment at end of the previous subsection. applies here as well} \yasaman{TO-DO address Jen's comments}

%\section{Categories of behavior likely to be impacted by anxiety and depression}
\paragraph{Passive Sensing of Mental and Physical Well-being:}
\label{sec:back-mhealth}
%%% OUTLINE %%%
% make sure to talk about the similar datasets (passive sensing of young adults) as you cover the following:

% \begin{itemize}
%     \item mobile sensing of physical health: sensors --> physical health markers
%     \item mobile sensing of mental health: sensors --> mental health markers
%     \item social media sensing of mental health: indicators --> mental health conditions
% \end{itemize}
Because discrimination is associated with anxiety and depression, we turn to literature on the impacts of these conditions on behavior for evidence of behavior types likely to be impacted by discrimination. 
Our review of the literature identified five primary categories of behavior, described below, each of which are linked indirectly or directly to depression in the literature. % \fig{fig:model} shows the changes in behavior predicted by the literature.

\paragraph{Physical Activity:}
\label{sec:back-mhealth-activity}
Higher levels of physical activity are correlated with fewer symptoms of anxiety and depression \cite{Stephens:1988} as well as lower levels of emotional distress \cite{Steptoe:1996}. Moreover, past work on mobile health sensing has successfully used features based on the inferred activity, \eg to predict depression in students \cite{Wang:2018} or relapse in schizophrenic patients \cite{Wang:2016}. We anticipate that exposure to discrimination leads to more depressed and anxious moods and it is thus negatively correlated with the levels of activity. We expect that following a discrimination encounter people become more sedentary. 
%\jm{This and each of the five should have a paralell construction: Introduce the concept and why it is relevant, talk about examples of past mobile sensing that has used this concept, talk about specific sensors that have been used or could be used to operationalize this concept}. \yasaman{I have covered your proposed structure except for sensors used and operationalization which is explained in \sect{sec:data}.} \yasaman{should we keep this sort of hypothesized change? our results are in opposite direction.}

\paragraph{Phone Usage:}
\label{sec:back-mhealth-phone}
Distraction, an emotion-regulation strategy to reduce distress and negative feelings \cite{Sheppes:2011}, can manifest itself in from of excessive or purposeless phone use. In fact, phone overuse is linked to depression and anxiety in college students \cite{Demirci:2015}. Encouragingly, patterns of phone use, particularly in relation to location of use, have been previously used as depression symptoms \cite{Wang:2018}. We therefore expect that exposure to discrimination is positively correlated with higher levels of phone use as people try lowering their distress through distraction.

\paragraph{Social Interactions:}
\label{sec:back-mhealth-socail}
Social support and interaction is key to psychological health and well-being \cite{Kawachi:2001}. Unsurprisingly, mental health problems, such as depression, are inversely related to quality and quantity of social interactions \cite{Nezlek:1994}. Moreover, social support seeking is a common strategy people use to cope with distress \cite{Carver:1997}. Operationalized in the form of phone calls, social interactions have been used as indicators of mental health \cite{Wang:2016}.   %In terms of social context, previous research has used information on nearby Bluetooth devices (\eg \cite{wang2014studentlife}), which can potentially encompass friendship information as suggested in \cite{Eagle:2009}.
Discriminatory encounters can initially lead to increased calls as people seek support. However, when the depressive symptoms (\eg withdrawal \cite{Girard:2014}) increase, social participation might drop. Records of phone calls can provide signals to this effect. %\jm{is this right?} \yasaman{I think it now makes better sense.}%\yasaman{TO-DO this has implications for analysis: (1) we can use calls / Bluetooth as a moderators that proxy social support. (2) we can also look at drop in social interactions}

\paragraph{Mobility:}
\label{sec:back-mhealth-mobility}
Mental health conditions, such as depression or anxiety, that are characterized by avoidance behaviors can potentially impact mobility patterns. There is indeed evidence connecting people's mobility to the severity of depressive symptoms \cite{Saeb:2015, Saeb:2016} and the levels of anxiety \cite{Huang:2016}. We anticipate that following discrimination experiences people's mobility patterns would change as their behavior is impacted by higher levels of depression and anxiety.
\paula{this is a comment}


\paragraph{Sleep:}
\label{sec:back-mhealth-sleep}
There is significant comorbidity between sleep problems and a number of mental health complications including depression \cite{Vandeputte:2003}, anxiety \cite{Morrison:1992}, and malconduct \cite{Papadimitriou:2005}. Sleep detection using wearable and mobile sensors has been the topic of mobile health research \cite{Min:2014} and commercially available, \eg through Fitbit. In relation to mobile sensing of mental health, \citet{wang2014studentlife} and \citet{Wang:2018} have used measures of sleep to model academic performance and levels of depression in college students. We expect disruptions to sleep follow discrimination encounters.

\vspace{1em}
\noindent
To summarize, the behaviors that we might expect based on the literature include reduced physical activity, increased phone screen time, reduced phone calls, reduced mobility, and increased sleep disruptions. However, the distance between the features of sensed behavior that can be passively collected and the data used in the studies cited is large. Our work will help to answer the question of whether these predicted differences are visible in global (study-long) \textit{vs} local (short-term) behavior of the sort sensed.

%\jm{we need some sort of summary here. This would also be a good place to put some words into convincing a CSCW person that they care, because this might impact CSCW style software?}

\section{Background and Related Work}
\label{sec:back}

\label{sec:back-discrimination}
\noindent 
Discrimination, defined as adverse behaviors, negative judgment, or unfair treatment towards a person, can be an uncontrollable and unpredictable stressor, and as such can severely influence health and well-being \citep{Williams:2009}. However, risk factors and protective factors often explain why similar experiences impact people differently \citep{SeridoAlmeidaWethington:2004}.  Stress is not evenly distributed. People with less privileged statuses have  greater exposure to stressors such as discrimination; further, stress proliferation increases their risk of subsequent cascades of stress exposure and  lessens protective resources, deepening negative impact and outcome disparities (Pearlin et al., 2005).

In terms of the responses it elicits and  health and functioning outcomes, the effect of discrimination can be studied within a stress and coping framework \citep{Pearlin:1999}. Ong \etal (\citeyear{Ong:2009}) apply this framework to  study  response to incidents of discrimination, using it to account for stress response variability based on the previous and current contexts of their lives. They discuss mechanisms through which chronic discrimination is linked to both daily discrimination as well as other negative events, which cumulatively lead to outcomes such as psychological distress. In a  study of  174 African American doctoral students, they demonstrated the relationship between both chronic and daily discrimination to distress; they showed that chronic exposure to discrimination exacerbates reactions to daily discrimination incidents, and that chronic and daily discrimination as well as other negative events each contribute to distress. Since then, other works have demonstrated a clear relationship between discrimination and stress (\eg \cite{Pieterse:2007}).  \paula{check my editing above; my effort here is to lay out some additional concepts that we will later use. Note I added a cite}
Because of discrimination's links to stress, it is not surprising that it can affect both physical and psychological well-being, as we now discuss.

\paragraph{Physical Well-Being:}
\label{sec:back-discrimination-physical}
When discrimination triggers a stress response, it can cause physiological changes such as heightened blood pressure, heart rate, and cortisol secretions \citep{Brondolo:2008, Steffen:2003, Smart:2010}); these can lead to serious conditions such as heart disease (\eg \cite{Marshall:1997, Cohen:1994}). Should it happen repeatedly, discrimination increases reactivitity to stressful situations \citep{GuyllMatthewsBrom-berger:2001} and weakens the body's protective resources, thus increasing the risk of illness similar to other forms of cumulative stress \citep{GeeSpencerChenTakeuchi:2007}. Additionally, discrimination is directly correlated with more unhealthy behavior (\eg smoking, drinking, substance use \citep{LandrineKlonoff:1996, MartinTuchRoman:2003}). 

\paragraph{Psychological Well-Being:}
\label{sec:back-discrimination-mental}
The association between exposure to discrimination and mental health is well supported by empirical evidence \citep{Pascoe:2009} as well as large-scale population studies \citep{Kessler:1999}. Not only is discrimination directly associated with higher levels of depression, anxiety, and psychological distress in general, it is negatively correlated with identifiers of healthy psyche, such as positive affect \citep{Schmitt:2014}. The magnitude of the associations is larger for negative health outcomes (\eg depression \citep{Schmitt:2014}) and is comparable to major stressors such as sexual assault or combat experience \citep{Kessler:1999}. 
Consistent with Ong \etal's (\citeyear{Ong:2009}) application of the stress and coping framework to discrimination, responses differ depending on factors like prior exposure  \citep{Kessler:1999}. 

\subsection{Open Questions}
\noindent The relationships documented in the literature are summarized in Figure~\ref{fig:model}.  A great deal of evidence shows that both daily and cumulative discrimination are linked to stress and thus impact physical and psychological well-being.  However, a unified model that relates short-term behavior to long-term outcomes still does not exist. Although Ong \etal (\citeyear{Ong:2009}) propose multiple models, their work focused only on self-reported measures. In addition, it is not yet clear how discrimination relates specifically to academic outcomes such as grades and retention. Finally, no prior work has quantified the impact of risk factors and protective factors on these relationships. A better understanding of the relationship between short-term changes and long-term outcomes can help us to design more effective interventions and improve our ability to model the overall effects of discrimination.